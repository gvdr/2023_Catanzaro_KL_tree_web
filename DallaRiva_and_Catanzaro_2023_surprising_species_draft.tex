%!TEX TS-program = xelatex
\documentclass[11pt]{article}

\usepackage[english]{babel}

\usepackage{amsmath,amssymb,amsfonts}
\usepackage[utf8]{inputenc}
\usepackage[T1]{fontenc}
\usepackage{stix2}
\usepackage[scaled]{helvet}
\usepackage[scaled]{inconsolata}

\usepackage{lastpage}

\usepackage{gensymb}

\usepackage{setspace}

\usepackage{ccicons}

\usepackage[hang,flushmargin]{footmisc}

\usepackage{geometry}

\setlength{\parindent}{0pt}
\setlength{\parskip}{6pt plus 2pt minus 1pt}

\usepackage{fancyhdr}
\renewcommand{\headrulewidth}{0pt}\providecommand{\tightlist}{%
  \setlength{\itemsep}{0pt}\setlength{\parskip}{0pt}}

\makeatletter
\newcounter{tableno}
\newenvironment{tablenos:no-prefix-table-caption}{
  \caption@ifcompatibility{}{
    \let\oldthetable\thetable
    \let\oldtheHtable\theHtable
    \renewcommand{\thetable}{tableno:\thetableno}
    \renewcommand{\theHtable}{tableno:\thetableno}
    \stepcounter{tableno}
    \captionsetup{labelformat=empty}
  }
}{
  \caption@ifcompatibility{}{
    \captionsetup{labelformat=default}
    \let\thetable\oldthetable
    \let\theHtable\oldtheHtable
    \addtocounter{table}{-1}
  }
}
\makeatother

\usepackage{array}
\newcommand{\PreserveBackslash}[1]{\let\temp=\\#1\let\\=\temp}
\let\PBS=\PreserveBackslash

\usepackage[breaklinks=true]{hyperref}
\hypersetup{colorlinks,%
citecolor=blue,%
filecolor=blue,%
linkcolor=blue,%
urlcolor=blue}
\usepackage{url}

\usepackage{caption}
\setcounter{secnumdepth}{0}
\usepackage{cleveref}

\usepackage{graphicx}
\makeatletter
\def\maxwidth{\ifdim\Gin@nat@width>\linewidth\linewidth
\else\Gin@nat@width\fi}
\makeatother
\let\Oldincludegraphics\includegraphics
\renewcommand{\includegraphics}[1]{\Oldincludegraphics[width=\maxwidth]{#1}}

\usepackage{longtable}
\usepackage{booktabs}

\usepackage{color}
\usepackage{fancyvrb}
\newcommand{\VerbBar}{|}
\newcommand{\VERB}{\Verb[commandchars=\\\{\}]}
\DefineVerbatimEnvironment{Highlighting}{Verbatim}{commandchars=\\\{\}}
% Add ',fontsize=\small' for more characters per line
\usepackage{framed}
\definecolor{shadecolor}{RGB}{248,248,248}
\newenvironment{Shaded}{\begin{snugshade}}{\end{snugshade}}
\newcommand{\KeywordTok}[1]{\textcolor[rgb]{0.13,0.29,0.53}{\textbf{#1}}}
\newcommand{\DataTypeTok}[1]{\textcolor[rgb]{0.13,0.29,0.53}{#1}}
\newcommand{\DecValTok}[1]{\textcolor[rgb]{0.00,0.00,0.81}{#1}}
\newcommand{\BaseNTok}[1]{\textcolor[rgb]{0.00,0.00,0.81}{#1}}
\newcommand{\FloatTok}[1]{\textcolor[rgb]{0.00,0.00,0.81}{#1}}
\newcommand{\ConstantTok}[1]{\textcolor[rgb]{0.00,0.00,0.00}{#1}}
\newcommand{\CharTok}[1]{\textcolor[rgb]{0.31,0.60,0.02}{#1}}
\newcommand{\SpecialCharTok}[1]{\textcolor[rgb]{0.00,0.00,0.00}{#1}}
\newcommand{\StringTok}[1]{\textcolor[rgb]{0.31,0.60,0.02}{#1}}
\newcommand{\VerbatimStringTok}[1]{\textcolor[rgb]{0.31,0.60,0.02}{#1}}
\newcommand{\SpecialStringTok}[1]{\textcolor[rgb]{0.31,0.60,0.02}{#1}}
\newcommand{\ImportTok}[1]{#1}
\newcommand{\CommentTok}[1]{\textcolor[rgb]{0.56,0.35,0.01}{\textit{#1}}}
\newcommand{\DocumentationTok}[1]{\textcolor[rgb]{0.56,0.35,0.01}{\textbf{\textit{#1}}}}
\newcommand{\AnnotationTok}[1]{\textcolor[rgb]{0.56,0.35,0.01}{\textbf{\textit{#1}}}}
\newcommand{\CommentVarTok}[1]{\textcolor[rgb]{0.56,0.35,0.01}{\textbf{\textit{#1}}}}
\newcommand{\OtherTok}[1]{\textcolor[rgb]{0.56,0.35,0.01}{#1}}
\newcommand{\FunctionTok}[1]{\textcolor[rgb]{0.00,0.00,0.00}{#1}}
\newcommand{\VariableTok}[1]{\textcolor[rgb]{0.00,0.00,0.00}{#1}}
\newcommand{\ControlFlowTok}[1]{\textcolor[rgb]{0.13,0.29,0.53}{\textbf{#1}}}
\newcommand{\OperatorTok}[1]{\textcolor[rgb]{0.81,0.36,0.00}{\textbf{#1}}}
\newcommand{\BuiltInTok}[1]{#1}
\newcommand{\ExtensionTok}[1]{#1}
\newcommand{\PreprocessorTok}[1]{\textcolor[rgb]{0.56,0.35,0.01}{\textit{#1}}}
\newcommand{\AttributeTok}[1]{\textcolor[rgb]{0.77,0.63,0.00}{#1}}
\newcommand{\RegionMarkerTok}[1]{#1}
\newcommand{\InformationTok}[1]{\textcolor[rgb]{0.56,0.35,0.01}{\textbf{\textit{#1}}}}
\newcommand{\WarningTok}[1]{\textcolor[rgb]{0.56,0.35,0.01}{\textbf{\textit{#1}}}}
\newcommand{\AlertTok}[1]{\textcolor[rgb]{0.94,0.16,0.16}{#1}}
\newcommand{\ErrorTok}[1]{\textcolor[rgb]{0.64,0.00,0.00}{\textbf{#1}}}
\newcommand{\NormalTok}[1]{#1}

\newlength{\cslhangindent}
\setlength{\cslhangindent}{1.5em}
\newlength{\csllabelwidth}
\setlength{\csllabelwidth}{3em}
\newenvironment{CSLReferences}[2] % #1 hanging-ident, #2 entry spacing
 {% don't indent paragraphs
  \setlength{\parindent}{0pt}
  % turn on hanging indent if param 1 is 1
  \ifodd #1 \everypar{\setlength{\hangindent}{\cslhangindent}}\ignorespaces\fi
  % set entry spacing
  \ifnum #2 > 0
  \setlength{\parskip}{#2\baselineskip}
  \fi
 }%
 {}
\usepackage{calc} % for \widthof, \maxof
\newcommand{\CSLBlock}[1]{#1\hfill\break}
\newcommand{\CSLLeftMargin}[1]{\parbox[t]{\maxof{\widthof{#1}}{\csllabelwidth}}{#1}}
\newcommand{\CSLRightInline}[1]{\parbox[t]{\linewidth}{#1}}
\newcommand{\CSLIndent}[1]{\hspace{\cslhangindent}#1}\geometry{verbose,letterpaper,tmargin=2.2cm,bmargin=2.2cm,lmargin=2.2cm,rmargin=2.2cm}

\usepackage{lineno}
\usepackage[nolists,noheads]{endfloat}

\pagestyle{plain}

\tolerance=1
\emergencystretch=\maxdimen
\hyphenpenalty=10000
\hbadness=10000

\doublespacing

\fancypagestyle{normal}
{
  \fancyhf{}
  \fancyfoot[R]{\footnotesize\sffamily\thepage\ of \pageref*{LastPage}}
}
\begin{document}
\raggedright
\thispagestyle{empty}
{\Large\bfseries\sffamily Surprising Species: the trace of evolution in
food webs through the lenses of information theory}
\vskip 5em

%
\href{https://orcid.org/0000-0002-0735-5184}{Giulio Valentino\,Dalla
Riva}%
%
\,\textsuperscript{1,2,‡}\quad %
Daniele\,Catanzaro%
%
\,\textsuperscript{3,4}

\textsuperscript{1}\,University of
Canterbury\quad \textsuperscript{2}\,School of Mathematics and
Statistics\quad \textsuperscript{3}\,Inn of the Prancing
Pony\quad \textsuperscript{4}\,Fellowship of the Ring

\textsuperscript{‡}\,Equal contributions\\

\textbf{Correspondance to:}\\
Giulio Valentino Dalla
Riva --- \texttt{giulio.dallariva@canterbury.ac.nz}\\

\vfill

Detecting and analysing the role of macro-evolutionary processes in the
structure of ecological networks is challenging, particularly for
unipartite food webs. Current methods are either model-based, or relying
on linear interactions, and provide limited information on the role of
single species. Here, we propose a novel approach, building on
Information Theory and providing a meaningful interpretation in terms of
Minimal Evolution, to test for phylogenetic signal in unipartite food
webs. Moreover, we show how that can be used to identify surprising
species: species that have an unexpected peculiar ecological role, given
their evolutionary history. We apply our methodology to two large food
webs, and discuss the results.




\vfill
This work is released by its authors under a CC-BY 4.0 license\hfill\ccby\\
Last revision: \emph{\today}

\clearpage
\thispagestyle{empty}

\vfill

\vfill

\clearpage
\linenumbers
\pagestyle{normal}

\hypertarget{introduction}{%
\section{Introduction}\label{introduction}}

Species communities, in their composition and structure, are the outcome
of both ecological and evolutionary processes. Darwin's remark about
\emph{endless forms most beautiful and most wonderful} can easily be
extended to food webs and the intricated nature of species interactions.
Indeed, the network nature of species communities is a costitutive
component of biodiversity. That is, different ecosystems differ not only
in the identity and abundance of species that compose them but, also, in
the web of mutual dependencies, cooperation, and competition that those
species draw.

Yet, because of the complexity of the information we deal with when we
handle networks, the network facet of biodiversity, and its dependency
on ecological and evolutionary processes, it's still largely unexplored.
Even the fundamental task of assessing how much the evolutionary history
of a community of species is reflected in the set of its ecological
interactions has not found a satisfactory methodological answer. In
particular, while the \emph{bipartite} case---where species can be
assigned to two distinct groups---has been treated in more extent,for
the \emph{unipartite} case---say, food webs---we don't have convincing
macro-evolutionary, mechanistic model of network temporal change (not a
consensus one, at least). The lack of sucha model, and even more the
lack of easily available data (that is, assembled phylogenies for food
webs), made it hard to explore the evolutionary signature of food webs.

It is, then, not surprising that more refined questions, such as a
stricter or loser adherence to the evolutionary signature of individual
species---that is, do species play the ecological role we expect them to
play given their evolutionary history?---are unanswered. We claim that
these questions, both at species and community leve, would enrich our
view of biodiversity, complementing assessment in terms of evolutionary
distinctiveness.

Here, we introduced a robust and computationally performing framework to
investigate the evolutionary signal of food webs, and assessing
individual species contribution, building on Information Theory and a
statistical model of complex networks, namely Random Dot Product Graphs.
Doing this, we introduced a notion of eco-evolutionary surprise: we
quantify how much additional information the observation of a food web
(and, at the species level, the realised interactions of a species) we
gain, if we already knew the evolutionary history of the species in the
food web.

Random Dot Product Graphs simplify the study of complex network by
representing them as low dimensional embeddings: nodes are mapped to
points in a pair of metric space---that is, geometric spaces where
pairwise distances are meaningful; interaction probability are then
encoded by distances between points. This framework is well established
in the statistical literature, and proved successfull for link
prediction in bipartite networks. The modelling of unipartite ecological
networks, and in particular food webs, through low dimensional
embeddings has shown to offer valuable insight for the ecologist. In
particular, Dalla Riva and Stouffer showed that XXX. Later, a fruitful
connection with phylogenetics was showcased by Strydom et al. In fact,
Strydom et al.~proved that ecological interaction information can be
transfered, through a common phylogeny, from the embedding of an
observed food web, to the embedding of an unobserved one, so predicting
whole communities food web structure. This latter result is underpinned
by a \emph{compatibility} of the network structure and phylogenetic
histories. More than that, the result was obtained by assuming a
simplistic model for the evolution of network interactions, namely, a
branching Brownian motion. That is, a model of evolution where lineages
evolve without any interaction: not a strong candidate for the evolution
of ecological interactions. Yet, the empirical success suggest the
presence of a signal strong enough to be detectable even under a wrongly
specified (but useful) model.

Information theoretical approaches are not new in phylogenetic.
Catanzaro et al.~showed that the tree reconstruction under the
assumptions of \emph{Balanced Minimum Evolution} corresponds to a
\emph{cross-entropy} minimization problem.

Building on these ideas, we quantify the surprise of a ecological
network given a phylogenetic tree as XXX. In addition, the surprise of a
species is given by the species contribution to the surprise of the
ecological network. And we interpret the surprise of a species as the
amount of additional information provided by the species ecological
interaction, given its phylogenetic history.

Here, we focus on unipartite food webs and phylogenetic trees. However,
the generalization to phylogenetic network, and the case of bipartite
webs is possible.

\hypertarget{methods}{%
\section{Methods}\label{methods}}

From phylogeny to distance matrix to distribution.

From ecological networks to distance matrix (through RDPG) to
distribution.

Compute Dkl of the two distributions.

Permanova by reshuffling tips on tree.

Decomposition on species level.

Figure 1: diagram of method.

\hypertarget{data}{%
\section{Data}\label{data}}

Tanzania National Park

Marine food webs

\hypertarget{results}{%
\section{Results}\label{results}}

Tanzania National Park

Tanzania Surprising Species

Figure 2: permanova test \textbar\textbar{} suprise of species
\textbar\textbar{} inlet of most suprising species.

Marine food webs

Marine surp spec

Figure 3: permanova test \textbar\textbar{} suprise of species
\textbar\textbar{} inlet of most suprising species

\hypertarget{conclusion}{%
\section{Conclusion}\label{conclusion}}

\hypertarget{references}{%
\section{References}\label{references}}

\end{document}
